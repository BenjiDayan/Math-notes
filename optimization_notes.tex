\documentclass{tufte-handout}
\usepackage{amsmath,amsfonts,amssymb, lipsum, enumitem}

% Set up the images/graphics package
\usepackage{graphicx}
\setkeys{Gin}{width=\linewidth,totalheight=\textheight,keepaspectratio}
\graphicspath{{figures/}}

\title{Benji Optimization Notes}
\author{Benjamin Dayan}

% Make prettier tables.
\usepackage{booktabs}

% The units package provides nice, non-stacked fractions and better spacing
% for units.
\usepackage{units}

% The fancyvrb package lets us customize the formatting of verbatim
% environments.  We use a slightly smaller font.
\usepackage{fancyvrb}
\fvset{fontsize=\normalsize}

% Small sections of multiple columns
\usepackage{multicol}

% control page layout / margins
\usepackage{geometry}
\geometry{
	a4paper,
	total={170mm,257mm},
	left=20mm,
	top=20mm,
	right=20mm,
}



%%% Custom Commands
%---------------------------------------------------------------------------
\newenvironment{defi}[1]
{\noindent \textbf{Definition: #1}\\
\itshape \noindent}
{\\}

\newenvironment{thm}[1]
{\noindent \textbf{Theorem: #1}\\
	\itshape \noindent}
{\newline}

\newenvironment{cor}[1]
{\noindent \textbf{Corollary: #1}\\
	\itshape \noindent}
{\\}

\setlist[enumerate]{itemsep=0mm}

% Concise referencing
\newcommand{\eqnref}[1]{\eqref{#1}}
\newcommand{\secref}[1]{Section \ref{#1}}
\newcommand{\figref}[1]{Figure \ref{#1}}
\newcommand{\lemref}[1]{Lemma \ref{#1}}
\newcommand{\corref}[1]{Corollary \ref{#1}}
\newcommand{\thmref}[1]{Theorem \ref{#1}}
% Real numbers
\newcommand{\Real}[1]{\mathbb{R}^{#1}}
% Complex numbers
\newcommand{\Complex}[1]{\mathbb{C}^{#1}}
% Integers
\newcommand{\Integer}[1]{\mathbb{Z}^{#1}}
% Rank operator
\DeclareMathOperator{\rank}{\textnormal{rank}}
% Vec operator
\newcommand{\vecop}{\textnormal{vec}}
% Norm
\newcommand{\norm}[1]{\left|\left|#1\right|\right|}
% Trace
\newcommand{\trace}{\textnormal{tr}}
% Range
\newcommand{\range}{\textnormal{range}}
% Partial derivative
\newcommand{\pd}[2]{\dfrac{\partial #1}{\partial #2}}
% Complete derivative
\newcommand{\dd}[2]{\dfrac{d #1}{d #2}}
% Complete derivative, second order
\newcommand{\dds}[2]{\dfrac{d^2 #1}{d {#2}^2}}
% Limit to N / N
\newcommand{\limover}[1]{\lim_{#1 \rightarrow \infty} \dfrac{1}{#1}}
% Display style sum
\newcommand{\dsum}{\displaystyle\sum}
% arg min and arg max
\newcommand{\argmax}[1]{\underset{#1}{\operatorname{arg~max}}}
\newcommand{\argmin}[1]{\underset{#1}{\operatorname{arg~min}}}
%---------------------------------------------------------------------------
\newtheorem{theorem}{Theorem}


\begin{document}
	
	\maketitle
	
	\section{Linear Programs}
	%---------------------------------------------------------------------------
	\begin{defi}{optimization under constraints standard form}
		minimize $f(x)$ subject to $g(x) = b$ and $x \in X$
	\end{defi}
	
	\begin{defi}{linear Program}
		minimize $c^\intercal x$ subject to $Ax = b$ and $x \geq 0$ hmm $c^\intercal x$
	\end{defi}

	\begin{defi}{functional, regional constraints $\implies$ feasible set}
	\end{defi}
	
	
	\begin{defi}{convex set}
	\end{defi}
	\begin{defi}{convex function}
	\end{defi}

	\begin{thm}{the feasible set of a LP is convex}
	\end{thm}

	\begin{defi}{extreme points}\end{defi}
	
	\begin{defi}{basic solutions and the basis}
		solutions to $Ax = b$ where $A$ is $m \times n$, and $x$ has at least $n-m$ zero variables. The \textbf{basis} is the $m \times m$ matrix in $A$, and is assumed to have linearly independent columns
	\end{defi}

	\begin{thm}{fundamental theorem of LP}
		\begin{enumerate}
			\item if there is a feasible solution, there is a basic feasible solution
			\item if there is an optimal feasible solution, there is an optimal basic feasible solution
		\end{enumerate}
	\end{thm}

	\begin{thm}{equivalence of basic feasible points and extreme points}
	
		
	
	
	
	\begin{theorem}
		This is a theorem
	\end{theorem}
	
	
	\section{Section 1}
	%---------------------------------------------------------------------------
	
	
	\subsection{Subsection 1}
	%---------------------------------------------------------------------------
\end{document}